%**************************************%
%*    Generated from PreTeXt source   *%
%*    on 2018-08-09T16:40:24Z    *%
%*                                    *%
%*   http://mathbook.pugetsound.edu   *%
%*                                    *%
%**************************************%
\documentclass[10pt,]{book}
%% Custom Preamble Entries, early (use latex.preamble.early)
%% Default LaTeX packages
%%   1.  always employed (or nearly so) for some purpose, or
%%   2.  a stylewriter may assume their presence
\usepackage{geometry}
%% Some aspects of the preamble are conditional,
%% the LaTeX engine is one such determinant
\usepackage{ifthen}
\usepackage{ifxetex,ifluatex}
%% Raster graphics inclusion
\usepackage{graphicx}
%% Colored boxes, and much more, though mostly styling
%% skins library provides "enhanced" skin, employing tikzpicture
%% boxes may be configured as "breakable" or "unbreakable"
%% "raster" controls grids of boxes, aka side-by-side
\usepackage{tcolorbox}
\tcbuselibrary{skins}
\tcbuselibrary{breakable}
\tcbuselibrary{raster}
%% xparse allows the construction of more robust commands,
%% this is a necessity for isolating styling and behavior
%% The tcolorbox library of the same name loads the base library
\tcbuselibrary{xparse}
%% Hyperref should be here, but likes to be loaded late
%%
%% Inline math delimiters, \(, \), need to be robust
%% 2016-01-31:  latexrelease.sty  supersedes  fixltx2e.sty
%% If  latexrelease.sty  exists, bugfix is in kernel
%% If not, bugfix is in  fixltx2e.sty
%% See:  https://tug.org/TUGboat/tb36-3/tb114ltnews22.pdf
%% and read "Fewer fragile commands" in distribution's  latexchanges.pdf
\IfFileExists{latexrelease.sty}{}{\usepackage{fixltx2e}}
%% Text height identically 9 inches, text width varies on point size
%% See Bringhurst 2.1.1 on measure for recommendations
%% 75 characters per line (count spaces, punctuation) is target
%% which is the upper limit of Bringhurst's recommendations
\geometry{letterpaper,total={340pt,9.0in}}
%% Custom Page Layout Adjustments (use latex.geometry)
%% This LaTeX file may be compiled with pdflatex, xelatex, or lualatex
%% The following provides engine-specific capabilities
%% Generally, xelatex and lualatex will do better languages other than US English
%% You can pick from the conditional if you will only ever use one engine
\ifthenelse{\boolean{xetex} \or \boolean{luatex}}{%
%% begin: xelatex and lualatex-specific configuration
%% fontspec package will make Latin Modern (lmodern) the default font
\ifxetex\usepackage{xltxtra}\fi
\usepackage{fontspec}
%% realscripts is the only part of xltxtra relevant to lualatex 
\ifluatex\usepackage{realscripts}\fi
%% 
%% Extensive support for other languages
\usepackage{polyglossia}
%% Main document language is US English
\setdefaultlanguage{english}
%% Spanish
\setotherlanguage{spanish}
%% Vietnamese
\setotherlanguage{vietnamese}
%% end: xelatex and lualatex-specific configuration
}{%
%% begin: pdflatex-specific configuration
%% translate common Unicode to their LaTeX equivalents
%% Also, fontenc with T1 makes CM-Super the default font
%% (\input{ix-utf8enc.dfu} from the "inputenx" package is possible addition (broken?)
\usepackage[T1]{fontenc}
\usepackage[utf8]{inputenc}
%% end: pdflatex-specific configuration
}
%% Symbols, align environment, bracket-matrix
\usepackage{amsmath}
\usepackage{amssymb}
%% allow page breaks within display mathematics anywhere
%% level 4 is maximally permissive
%% this is exactly the opposite of AMSmath package philosophy
%% there are per-display, and per-equation options to control this
%% split, aligned, gathered, and alignedat are not affected
\allowdisplaybreaks[4]
%% allow more columns to a matrix
%% can make this even bigger by overriding with  latex.preamble.late  processing option
\setcounter{MaxMatrixCols}{30}
%%
%% Color support, xcolor package
%% Always loaded, for: add/delete text, author tools
\PassOptionsToPackage{usenames,dvipsnames,svgnames,table}{xcolor}
\usepackage{xcolor}
%%
%% Semantic Macros
%% To preserve meaning in a LaTeX file
%% Only defined here if required in this document
%% Subdivision Numbering, Chapters, Sections, Subsections, etc
%% Subdivision numbers may be turned off at some level ("depth")
%% A section *always* has depth 1, contrary to us counting from the document root
%% The latex default is 3.  If a larger number is present here, then
%% removing this command may make some cross-references ambiguous
%% The precursor variable $numbering-maxlevel is checked for consistency in the common XSL file
\setcounter{secnumdepth}{3}
%% begin: General AMS environment setup
%% Environments built with amsthm package
\usepackage{amsthm}
%% Numbering for Theorems, Conjectures, Examples, Figures, etc
%% Controlled by  numbering.theorems.level  processing parameter
%% Numbering: all theorem-like numbered consecutively
%% i.e. Corollary 4.3 follows Theorem 4.2
%% Always need some theorem environment to set base numbering scheme
%% even if document has no theorems (but has other environments)
%% Create a never-used style first, always
%% simply to provide a global counter to use, namely "cthm"
\newtheorem{cthm}{BadTheoremStringName}[section]
%% end: General AMS environment setup
%% begin: environments with italicized bodies, theorems and similar
%% Style is like a theorem, and for statements without proofs
%% Theorem-like environments in modified "plain" style
%% We manage the head, do not adjust vertical spacing
%% Thus "space after theorem head" is necessary
%% This provides an automatic period after the number
\newtheoremstyle{ptxplainnotitle}
  {}% space above
  {}% space below
  {\itshape}% body font
  {}% indent amount
  {\bfseries}% theorem head font
  {.}% punctuation after theorem head
  {0.5em}% space after theorem head
  {}% theorem head specification
%% We now manage punctuation on-sight, elsewhere,
%% assuming non-trivial content inside a "title"
\newtheoremstyle{ptxplaintitle}
  {}% space above
  {}% space below
  {\itshape}% body font
  {}% indent amount
  {\bfseries}% theorem head font
  {}% punctuation after theorem head
  {0.5em}% space after theorem head
  {\thmname{#1}\thmnumber{ #2}\thmnote{ #3}}% theorem head specification
%% Only variants actually used in document appear here
%% Template eventually creates an environment of the given name
%% No arguments => Theorem 2.6. via "notitle" style variant
%% One optional argument => Theorem 2.6 Fantastic! via "title" style variant
\theoremstyle{ptxplainnotitle}
\newtheorem{factnotitle}[cthm]{Fact}
\theoremstyle{ptxplaintitle}
\newtheorem{facttitle}[cthm]{Fact}
\NewDocumentEnvironment{fact}{ot+o}
    { \IfBooleanTF{#2}
        { \IfValueTF{#1}{ \begin{facttitle}[#3 {\mdseries(#1)}] }{ \begin{facttitle}[#3]} } 
        { \IfValueTF{#1}{ \begin{factnotitle}[#1]  }{ \begin{factnotitle}} }
    }
    { \IfBooleanTF{#2}{\end{facttitle}}{\end{factnotitle}} }
%% AMS proof environment is basically fine as-is and special treatment
%% would certainly interfere with the functioning of \qed, etc.
%% So we simply localize the default heading
%% Redefinition of the "proof" environment is to cause a long alternate
%% title to line-break appropriately.  Code is cut verbatim, by suggestion,
%% from "Using the amsthm Package" Version 2.20.3, September 2017
%% end: environments with italicized bodies, theorems and similar
%% begin: environments with normal bodies, examples, etc.
%% Other environments go in modified "definition" style
%% Similar to above
\newtheoremstyle{ptxdefinitionnotitle}
  {}% space above
  {}% space below
  {}% body font
  {}% indent amount
  {\bfseries}% theorem head font
  {.}% punctuation after theorem head
  {0.5em}% space after theorem head
  {\thmname{#1}\thmnumber{ #2}}% theorem head specification
\newtheoremstyle{ptxdefinitiontitle}
  {}% space above
  {}% space below
  {}% body font
  {}% indent amount
  {\bfseries}% theorem head font
  {}% punctuation after theorem head
  {0.5em}% space after theorem head
  {\thmname{#1}\thmnumber{ #2}\thmnote{ #3}}% theorem head specification
\theoremstyle{ptxdefinitionnotitle}
\newtheorem{definitionnotitle}[cthm]{Definition}
\theoremstyle{ptxdefinitiontitle}
\newtheorem{definitiontitle}[cthm]{Definition}
\NewDocumentEnvironment{definition}{o}
  {\IfValueTF{#1}{\begin{definitiontitle}[{#1}]}{\begin{definitionnotitle}}}
  {\IfValueTF{#1}{\end{definitiontitle}}{\end{definitionnotitle}}}
\theoremstyle{ptxdefinitionnotitle}
\newtheorem{notenotitle}[cthm]{Note}
\theoremstyle{ptxdefinitiontitle}
\newtheorem{notetitle}[cthm]{Note}
\NewDocumentEnvironment{note}{o}
  {\IfValueTF{#1}{\begin{notetitle}[{#1}]}{\begin{notenotitle}}}
  {\IfValueTF{#1}{\end{notetitle}}{\end{notenotitle}}}
\theoremstyle{ptxdefinitionnotitle}
\newtheorem{examplenotitle}[cthm]{Example}
\theoremstyle{ptxdefinitiontitle}
\newtheorem{exampletitle}[cthm]{Example}
\NewDocumentEnvironment{example}{o}
  {\IfValueTF{#1}{\begin{exampletitle}[{#1}]}{\begin{examplenotitle}}}
  {\IfValueTF{#1}{\end{exampletitle}}{\end{examplenotitle}}}
%% end: environments with normal bodies, examples, etc.
%% begin: environments for project-like, with independent counter
%% Numbering for Projects (independent of others)
%% Controlled by  numbering.projects.level  processing parameter
%% Always need a project environment to set base numbering scheme
%% even if document has no projectss (but has other blocks)
%% So "cpjt" environment produces "cpjt" counter
\newtheorem{cpjt}{BadProjectNameString}[section]
\theoremstyle{ptxdefinitionnotitle}
\newtheorem{activitynotitle}[cpjt]{Activity}
\theoremstyle{ptxdefinitiontitle}
\newtheorem{activitytitle}[cpjt]{Activity}
\NewDocumentEnvironment{activity}{o}
  {\IfValueTF{#1}{\begin{activitytitle}[{#1}]}{\begin{activitynotitle}}}
  {\IfValueTF{#1}{\end{activitytitle}}{\end{activitynotitle}}}
%% end: environments for project-like, with independent counter
%% Localize LaTeX supplied names (possibly none)
\renewcommand*{\chaptername}{Chapter}
%% More flexible list management, esp. for references
%% But also for specifying labels (i.e. custom order) on nested lists
\usepackage{enumitem}
%% hyperref driver does not need to be specified, it will be detected
\usepackage{hyperref}
%% Hyperlinking active in PDFs, all links solid and blue
\hypersetup{colorlinks=true,linkcolor=blue,citecolor=blue,filecolor=blue,urlcolor=blue}
\hypersetup{pdftitle={Team-Based Inquiry Learning - Linear Algebra}}
%% If you manually remove hyperref, leave in this next command
\providecommand\phantomsection{}
%% If tikz has been loaded, replace ampersand with \amp macro
%% extpfeil package for certain extensible arrows,
%% as also provided by MathJax extension of the same name
%% NB: this package loads mtools, which loads calc, which redefines
%%     \setlength, so it can be removed if it seems to be in the 
%%     way and your math does not use:
%%     
%%     \xtwoheadrightarrow, \xtwoheadleftarrow, \xmapsto, \xlongequal, \xtofrom
%%     
%%     we have had to be extra careful with variable thickness
%%     lines in tables, and so also load this package late
\usepackage{extpfeil}
%% Custom Preamble Entries, late (use latex.preamble.late)
%% Begin: Author-provided packages
%% (From  docinfo/latex-preamble/package  elements)
%% End: Author-provided packages
%% Begin: Author-provided macros
%% (From  docinfo/macros  element)
%% Plus three from MBX for XML characters
\newcommand{\IR}{\mathbb{R}}
\newcommand{\Poly}{\mathcal{P}}
\newcommand{\doubler}[1]{2#1}
\newcommand{\lt}{<}
\newcommand{\gt}{>}
\newcommand{\amp}{&}
%% End: Author-provided macros
\begin{document}
\frontmatter
%% begin: half-title
\thispagestyle{empty}
{\centering
\vspace*{0.28\textheight}
{\Huge Team-Based Inquiry Learning - Linear Algebra}\\}
\clearpage
%% end:   half-title
%% begin: adcard
\thispagestyle{empty}
\null%
\clearpage
%% end:   adcard
%% begin: title page
%% Inspired by Peter Wilson's "titleDB" in "titlepages" CTAN package
\thispagestyle{empty}
{\centering
\vspace*{0.14\textheight}
%% Target for xref to top-level element is ToC
\addtocontents{toc}{\protect\hypertarget{tbil-example}{}}
{\Huge Team-Based Inquiry Learning - Linear Algebra}\\[3\baselineskip]
{\Large Steven Clontz \& Drew Lewis}\\[0.5\baselineskip]
{\Large University of South Alabama}\\[3\baselineskip]
{\Large August 9, 2018}\\}
\clearpage
%% end:   title page
%% begin: copyright-page
\thispagestyle{empty}
\vspace*{\stretch{2}}
\vspace*{\stretch{1}}
\null\clearpage
%% end:   copyright-page
\hypertarget{p-1}{}%
This is a port of the first standard covered in Clontz and Lewis's Team-Based Inquiry Learning module on Linear Transformations.%
%% begin: table of contents
%% Adjust Table of Contents
\setcounter{tocdepth}{1}
\renewcommand*\contentsname{Contents}
\tableofcontents
%% end:   table of contents
\mainmatter
\hypertarget{p-2}{}%
A linear transformation (or linear map) is a rule that describes how to transform vectors in one vector space into vectors in another vector space. Here I am adding a ``quote'' just for practice.%
\typeout{************************************************}
\typeout{Chapter 1 Module A: Algebra of Linear Transformations}
\typeout{************************************************}
\chapter[{Module A: Algebra of Linear Transformations}]{Module A: Algebra of Linear Transformations}\label{chapter-1}
\typeout{************************************************}
\typeout{Section 1.1 Determine if a map between vector spaces of polynomials is linear or not.}
\typeout{************************************************}
\section[{Determine if a map between vector spaces of polynomials is linear or not.}]{Determine if a map between vector spaces of polynomials is linear or not.}\label{section-standard-A1}
\hypertarget{p-3}{}%
This is where I can write a short introductory slide (or not) to the section.%
\begin{definition}\label{definition-1}
\hypertarget{p-4}{}%
A \emph{linear transformation} (also known as a \emph{linear map}) is a map between vector spaces that preserves the vector space operations.%
\par
\hypertarget{p-5}{}%
More precisely, if \(V\) and \(W\) are vector spaces, a map \(T:V\rightarrow W\) is called a linear transformation if%
\leavevmode%
\begin{enumerate}
\item\hypertarget{li-1}{}\(T(\vec{v}+\vec{w}) = T(\vec{v})+T(\vec{w})\) for any \(\vec{v},\vec{w} \in V\).%
\item\hypertarget{li-2}{}\(T(c\vec{v}) = cT(\vec{v})\) for any \(c \in \IR,\vec{v} \in V\).%
\end{enumerate}
\hypertarget{p-6}{}%
In other words, a map is linear when vector space operations can be applied before or after the transformation without affecting the result.%
\end{definition}
\begin{example}\label{example-1}
\hypertarget{p-7}{}%
Let \(T : \IR^3 \rightarrow \IR^2\) be given by%
%
\begin{equation*}
T\left(\begin{bmatrix} x \\ y \\ z \end{bmatrix} \right)
=
\begin{bmatrix} x-z \\ 3y \end{bmatrix}
\end{equation*}
\hypertarget{p-8}{}%
To show that \(T\) is linear, we must verify...%
%
\begin{equation*}
T\left(
\begin{bmatrix} x \\ y \\ z \end{bmatrix} +
\begin{bmatrix} u \\ v \\ w \end{bmatrix}
\right)
=
T\left(
\begin{bmatrix} x+u \\ y+v \\ z+w \end{bmatrix}
\right) =
\begin{bmatrix} (x+u)-(z+w) \\ 3(y+v) \end{bmatrix}
\end{equation*}
%
\begin{equation*}
T\left(
\begin{bmatrix} x \\ y \\ z \end{bmatrix}
\right) + T\left(
\begin{bmatrix} u \\ v \\ w \end{bmatrix}
\right)
=
\begin{bmatrix} x-z \\ 3y \end{bmatrix} +
\begin{bmatrix} u-w \\ 3v \end{bmatrix}=
\begin{bmatrix} (x+u)-(z+w) \\ 3(y+v) \end{bmatrix}
\end{equation*}
\hypertarget{p-9}{}%
And also...%
%
\begin{equation*}
T\left(c\begin{bmatrix} x \\ y \\ z \end{bmatrix} \right)
=
T\left(\begin{bmatrix} cx \\ cy \\ cz \end{bmatrix} \right)
=
\begin{bmatrix} cx-cz \\ 3cy \end{bmatrix}
\text{ and }
cT\left(\begin{bmatrix} x \\ y \\ z \end{bmatrix} \right)
=
c\begin{bmatrix} x-z \\ 3y \end{bmatrix}
=
\begin{bmatrix} cx-cz \\ 3cy \end{bmatrix}
\end{equation*}
\hypertarget{p-10}{}%
Therefore \(T\) is a linear transformation.%
\end{example}
\begin{example}\label{example-2}
\hypertarget{p-11}{}%
Let \(T : \IR^2 \rightarrow \IR^4\) be given by%
%
\begin{equation*}
T\left(\begin{bmatrix} x \\ y \end{bmatrix} \right)
=
\begin{bmatrix} x+y \\ x^2 \\ y+3 \\ y-2^x \end{bmatrix}
\end{equation*}
\hypertarget{p-12}{}%
To show that \(T\) is not linear, we only need to find one counterexample.%
%
\begin{equation*}
T\left(
\begin{bmatrix} 0 \\ 1 \end{bmatrix} +
\begin{bmatrix} 2 \\ 3 \end{bmatrix}
\right)
=
T\left(
\begin{bmatrix} 2 \\ 4 \end{bmatrix}
\right) =
\begin{bmatrix} 6 \\ 4 \\ 7 \\ 0 \end{bmatrix}
\end{equation*}
%
\begin{equation*}
T\left(
\begin{bmatrix} 0 \\ 1 \end{bmatrix}
\right) + T\left(
\begin{bmatrix} 2 \\ 3\end{bmatrix}
\right)
=
\begin{bmatrix} 1 \\ 0 \\ 4 \\ -1 \end{bmatrix} +
\begin{bmatrix} 5 \\ 4 \\ 6 \\ -5 \end{bmatrix}
=
\begin{bmatrix} 6 \\ 4 \\ 10 \\ -6 \end{bmatrix}
\end{equation*}
\hypertarget{p-13}{}%
Since the resulting vectors are different, \(T\) is not a linear transformation.%
\end{example}
\begin{fact}\label{fact-1}
\hypertarget{p-14}{}%
A map between Euclidean spaces \(T:\IR^n\to\IR^m\) is linear exactly when every component of the output is a linear combination of the components of \(\IR^n\).%
\par
\hypertarget{p-15}{}%
For example, the following map is definitely linear because \(x-z\) and \(3y\) are linear combinations of \(x,y,z\):%
%
\begin{equation*}
T\left(\begin{bmatrix} x \\ y \\ z \end{bmatrix} \right)
=
\begin{bmatrix} x-z \\ 3y \end{bmatrix}
=
\begin{bmatrix} 1x+0y-1z \\ 0x+3y+0z \end{bmatrix}
\end{equation*}
\hypertarget{p-16}{}%
But this map is not linear because \(x^2\), \(y+3\), and \(y-2^x\) are not linear combinations (even though \(x+y\) is):%
%
\begin{equation*}
T\left(\begin{bmatrix} x \\ y \end{bmatrix} \right)
=
\begin{bmatrix} x+y \\ x^2 \\ y+3 \\ y-2^x \end{bmatrix}
\end{equation*}
\end{fact}
\begin{activity}\label{activity-1}
\hypertarget{p-17}{}%
\emph{(~5 min)}%
\par
\hypertarget{p-18}{}%
Recall the following rules from calculus, where \(D:\Poly\to\Poly\) is the derivative map defined by \(D(f(x))=f'(x)\) for each polynomial \(f\).%
%
\begin{equation*}
D(f+g)=f'(x)+g'(x)
\end{equation*}
%
\begin{equation*}
D(cf(x))=cf'(x)
\end{equation*}
\hypertarget{p-19}{}%
What can we conclude from these rules?%
\leavevmode%
\begin{enumerate}[label=\Alph*]
\item\hypertarget{li-3}{}\(\Poly\) is not a vector space%
\item\hypertarget{li-4}{}\(D\) is a linear map%
\item\hypertarget{li-5}{}\(D\) is not a linear map%
\end{enumerate}
\end{activity}
\begin{activity}\label{activity-2}
\hypertarget{p-20}{}%
\emph{(~10 min)}%
\par
\hypertarget{p-21}{}%
Let the polynomial maps \(S: \Poly^4 \rightarrow \Poly^3\) and \(T: \Poly^4 \rightarrow \Poly^3\) be defined by%
%
\begin{equation*}
S(f(x)) = 2f'(x)-f''(x) \hspace{3em} T(f(x)) = f'(x)+x^3
\end{equation*}
\hypertarget{p-22}{}%
Compute \(S(x^4+x)\), \(S(x^4)+S(x)\), \(T(x^4+x)\), and \(T(x^4)+T(x)\). Which of these maps is definitely not linear?%
\end{activity}
\begin{fact}\label{fact-2}
\hypertarget{p-23}{}%
If \(L:V\to W\) is linear, then%
\begin{equation*}
L(\vec z)=L(0\vec v)=0L(\vec v)=\vec z
\end{equation*}
where \(\vec z\) is the additive identity of the vector spaces \(V,W\).%
\par
\hypertarget{p-24}{}%
Put another way, an easy way to prove that a map like \(T(f(x)) = f'(x)+x^3\) can't be linear is because%
\begin{equation*}
T(0)=\frac{d}{dx}[0]+x^3=0+x^3=x^3\not=0.
\end{equation*}
%
\end{fact}
\begin{note}\label{note-1}
\hypertarget{p-25}{}%
Showing \(L:V\to W\) is not a linear transformation can be done by finding an example for any one of the following.%
\leavevmode%
\begin{itemize}[label=\textbullet]
\item{}Show \(L(\vec z)\not=\vec z\) (where \(\vec z\) is the additive identity of \(L\) and \(W\)).%
\item{}Find \(\vec v,\vec w\in V\) such that \(L(\vec v+\vec w)\not=L(\vec v)+L(\vec w)\).%
\item{}Find \(\vec v\in V\) and \(c\in \IR\) such that \(L(c\vec v)\not=cL(\vec v)\).%
\end{itemize}
\hypertarget{p-26}{}%
Otherwise, \(L\) can be shown to be linear by proving the following in general.%
\leavevmode%
\begin{itemize}[label=\textbullet]
\item{}For all \(\vec v,\vec w\in V\), \(L(\vec v+\vec w)\not=L(\vec v)+L(\vec w)\).%
\item{}For all \(\vec v\in V\) and \(c\in \IR\), \(L(c\vec v)\not=cL(\vec v)\).%
\end{itemize}
\hypertarget{p-27}{}%
Note the similarities between this process and showing that a subset of a vector space is/isn't a subspace.%
\end{note}
\begin{activity}\label{activity-3}
\hypertarget{p-28}{}%
\emph{(~15 min)}%
\par
\hypertarget{p-29}{}%
Continue to consider \(S: \Poly^4 \rightarrow \Poly^3\) defined by%
%
\begin{equation*}
S(f(x)) = 2f'(x)-f''(x)
\end{equation*}
\begin{enumerate}[font=\bfseries,label=(\alph*),ref=\alph*]
\item\label{task-1} \hypertarget{p-30}{}%
Verify that%
\begin{equation*}
S(f(x)+g(x))=2f'(x)+2g'(x)-f''(x)-g''(x)
\end{equation*}
is equal to \(S(f(x))+S(g(x))\) for all polynomials \(f,g\).%
\item\label{task-2} \hypertarget{p-31}{}%
Verify that \(S(cf(x))\) is equal to \(cS(f(x))\) for all real numbers \(c\) and polynomials \(f\).%
\item\label{task-3} \hypertarget{p-32}{}%
Is \(S\) linear?%
\end{enumerate}
\end{activity}
\begin{activity}\label{activity-4}
\hypertarget{p-33}{}%
\emph{(~20 min)}%
\par
\hypertarget{p-34}{}%
Let the polynomial maps \(S: \Poly \rightarrow \Poly\) and \(T: \Poly \rightarrow \Poly\) be defined by%
%
\begin{equation*}
S(f(x)) = (f(x))^2 \hspace{3em} T(f(x)) = 3xf(x^2)
\end{equation*}
\begin{enumerate}[font=\bfseries,label=(\alph*),ref=\alph*]
\item\label{task-4} \hypertarget{p-35}{}%
Note that \(S(0)=0\) and \(T(0)=0\). So instead, show that \(S(x+1)\not= S(x)+S(1)\) to verify that \(S\) is not linear.%
\item\label{task-5} \hypertarget{p-36}{}%
Prove that \(T\) is linear by verifying that \(T(f(x)+g(x))=T(f(x))+T(g(x))\) and \(T(cf(x))=cT(f(x))\).%
\end{enumerate}
\end{activity}
\end{document}